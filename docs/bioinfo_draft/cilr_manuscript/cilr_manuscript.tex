\documentclass{bioinfo}
\copyrightyear{2016} \pubyear{2016}

%% Some pieces required from the pandoc template
\providecommand{\tightlist}{%
  \setlength{\itemsep}{0pt}\setlength{\parskip}{0pt}}


% Pandoc citation processing


% hyperref makes the margins screwy.
% https://groups.google.com/forum/#!topic/latexusersgroup/4W_SwGk6zx4
% http://ansuz.sooke.bc.ca/software/latex-tricks.php
% \usepackage[colorlinks=true, allcolors=blue]{hyperref}

\access{Advance Access Publication Date:   }
\appnotes{Manuscript Category}

\begin{document}
\firstpage{1}

\subtitle{Original Paper}

\title[Bioinformatics Rmd Template]{Template for preparing a submission
to Bioinformatics using RMarkdown}

\author[FirstAuthorLastName \textit{et~al}.]{
Quang P. Nguyen\,\textsuperscript{1,2},
Anne G. Hoen\,\textsuperscript{1,2},
H. Robert Frost\,\textsuperscript{2},
}

\address{
\textsuperscript{1}Department of Epidemiology, Geisel School of Medicine
at Dartmouth College\\
\textsuperscript{2}Department of Biomedical Data Science, Geisel School
of Medicine at Dartmouth College\\
}

\corresp{To whom correspondence should be addressed. E-mail:
bob@email.com}

\history{Received on XXX; revised on XXX; accepted on XXX}

\editor{Associate Editor: XXX}

\abstract{
\textbf{Motivation:} This section should specifically state the
scientific question within the context of the field of study.\\
\textbf{Results:} This section should summarize the scientific advance
or novel results of the study, and its impact on computational
biology.\\
\textbf{Availability:} This section should state software availability
if the paper focuses mainly on software development or on the
implementation of an algorithm. Examples are: `Freely available on the
web at XXX.' Website implemented in Perl, MySQL and Apache, with all
major browsers supported'; or `Source code and binaries freely available
for download at URL, implemented in C++ and supported on linux and MS
Windows'. The complete address (URL) should be given. If the manuscript
describes new software tools or the implementation of novel algorithms
the software must be freely available to non-commercial users. Authors
must also ensure that the software is available for a full TWO YEARS
following publication. The editors of Bioinformatics encourage authors
to make their source code available and, if possible, to provide access
through an open source license.\\
\textbf{Contact:}bob@email.com\\
\textbf{Supplementary information:} Supplementary data are available at
Bioinformatics Online.}

\maketitle

\section{Introduction}

\section{Materials and Methods}

\section{Methods}

\subsection{Competitive Isometric Log-ratio (cILR)}

The cILR method generates sample-specific enrichment scores for
microbial sets using the isometric log-ratio transformation
\citep{egozcue2003}. The cILR method takes two inputs:

\begin{itemize}
\tightlist
\item
  \(\mathbf{X}\): \(n\) by \(p\) matrix of positive counts for \(p\)
  taxa and \(n\) samples measured through either targeted sequencing
  \(p\) matrix of positive counts for \(p\) taxa and \(n\) samples
  measured through either targeted sequencing (such as 16S rRNA) or
  whole genome shotgun sequencing. Usually \(\mathbf{X}\) is generated
  from standard sequence processing pipelines such as DADA2
  \citep{callahan2016} and MetaPhlAn2 \citep{truong2015}.
\item
  \(\mathbf{A}\): \(p\) by \(m\) indicator matrix annotation the
  membership of each taxa \(p\) to \(m\) sets of interest. These sets
  can be Linnean taxonomic classifications annotated using databases
  such as SILVA \citep{quast2013} or those based on more functionally
  driven categories such as the functional tropism of microbes
  (\(a_{i,j} = 1\) indicates that microbe \(i\) belongs to set \(j\)
\end{itemize}

The cILR method generates one output:

\begin{itemize}
\tightlist
\item
  \(\mathbf{E}\): \(n\) by \(m\) matrix indicating the enrichment score
  of \(m\) pre-defined sets identified in
\end{itemize}

\begin{itemize}
    \item $\mathbf{E}$: $n$ by $m$ matrix indicating the enrichment score of $m$ pre-defined sets identified in $\mathbf{A}$ across $n$ samples. 
\end{itemize}

The procedure is as follows:\\

\begin{enumerate}
    \item \textbf{Compute the cILR statistic}: Let $\mathbf{M}$ be a $n$ by $m$ matrix of cILR scores. Let $\mathbf{M}_{i,k}$ be cILR scores for set $k$ of sample $i$:   
    $$\mathbf{M}_{i,k} = \sqrt{\frac{\sum_k A_{ik}(p - \sum_k A_{ik})}{p}} \ln \left( \frac{g(\mathbf{X}_{i,j}|\mathbf{A}_{j,k} = 1)}{g(\mathbf{X}_{i,j}|\mathbf{A}_{j,k} \neq 1))} \right)$$
    where $g()$ is the geometric mean. This represents the ratio of the geometric mean of the relative abundance of taxa assigned to set $k$ and remainder taxa.   
    \item \textbf{Compute the cILR statistic on permuted $\mathbf{X}$}: We seek to evaluate the empirical null distribution of the cILR statistic under $H_o$ that relative abundances in $\mathbf{X}$ of members of set $k$ are not enriched compared to those not in set $k$. Let $\mathbf{X}_p$ be the column permuted relative abundance matrix, and $\mathbf{M}_p$ be the corresponding cILR scores generated from $\mathbf{X}_p$. 
    \item \textbf{Fit Gaussian mixture distribution for each column of $\mathbf{M}_p$}
    \item \textbf{Calculate finalized cILR scores as CDF values of the fitted mixture distribution}
\end{enumerate}

Details for Method 1. Lorem ipsum ad nauseum. Introduce your topic.
Lorem ipsum ad nauseum. Introduce your topic. Lorem ipsum ad nauseum.
Introduce your topic. Lorem ipsum ad nauseum. Introduce your topic.
Lorem ipsum ad nauseum.

Details for Method 1. Lorem ipsum ad nauseum. Introduce your topic.
Lorem ipsum ad nauseum. Introduce your topic. Lorem ipsum ad nauseum.
Introduce your topic. Lorem ipsum ad nauseum. Introduce your topic.
Lorem ipsum ad nauseum.

Details for Method 1. Lorem ipsum ad nauseum. Introduce your topic.
Lorem ipsum ad nauseum. Introduce your topic. Lorem ipsum ad nauseum.
Introduce your topic. Lorem ipsum ad nauseum. Introduce your topic.
Lorem ipsum ad nauseum.

Details for Method 1. Lorem ipsum ad nauseum. Introduce your topic.
Lorem ipsum ad nauseum. Introduce your topic. Lorem ipsum ad nauseum.
Introduce your topic. Lorem ipsum ad nauseum. Introduce your topic.
Lorem ipsum ad nauseum.

\subsection{Method 2}

Details for Method 2. Lorem ipsum ad nauseum. Introduce your topic.
Lorem ipsum ad nauseum. Introduce your topic. Lorem ipsum ad nauseum.
Introduce your topic. Lorem ipsum ad nauseum. Introduce your topic.
Lorem ipsum ad nauseum.

Details for Method 2. Lorem ipsum ad nauseum. Introduce your topic.
Lorem ipsum ad nauseum. Introduce your topic. Lorem ipsum ad nauseum.
Introduce your topic. Lorem ipsum ad nauseum. Introduce your topic.
Lorem ipsum ad nauseum.

Details for Method 2. Lorem ipsum ad nauseum. Introduce your topic.
Lorem ipsum ad nauseum. Introduce your topic. Lorem ipsum ad nauseum.
Introduce your topic. Lorem ipsum ad nauseum. Introduce your topic.
Lorem ipsum ad nauseum.

Details for Method 2. Lorem ipsum ad nauseum. Introduce your topic.
Lorem ipsum ad nauseum. Introduce your topic. Lorem ipsum ad nauseum.
Introduce your topic. Lorem ipsum ad nauseum. Introduce your topic.
Lorem ipsum ad nauseum.

\section{Discussion}

Discussion of results. Lorem ipsum ad nauseum. Introduce your topic.
Lorem ipsum ad nauseum. Introduce your topic. Lorem ipsum ad nauseum.
Introduce your topic. Lorem ipsum ad nauseum. Introduce your topic.
Lorem ipsum ad nauseum.

Discussion of results. Lorem ipsum ad nauseum. Introduce your topic.
Lorem ipsum ad nauseum. Introduce your topic. Lorem ipsum ad nauseum.
Introduce your topic. Lorem ipsum ad nauseum. Introduce your topic.
Lorem ipsum ad nauseum.

Discussion of results. Lorem ipsum ad nauseum. Introduce your topic.
Lorem ipsum ad nauseum. Introduce your topic. Lorem ipsum ad nauseum.
Introduce your topic. Lorem ipsum ad nauseum. Introduce your topic.
Lorem ipsum ad nauseum.

Discussion of results. Lorem ipsum ad nauseum. Introduce your topic.
Lorem ipsum ad nauseum. Introduce your topic. Lorem ipsum ad nauseum.
Introduce your topic. Lorem ipsum ad nauseum. Introduce your topic.
Lorem ipsum ad nauseum.

\section{Conclusion}

Anything else? Lorem ipsum ad nauseum. Introduce your topic. Lorem ipsum
ad nauseum. Introduce your topic. Lorem ipsum ad nauseum. Introduce your
topic. Lorem ipsum ad nauseum. Introduce your topic. Lorem ipsum ad
nauseum.

Anything else? Lorem ipsum ad nauseum. Introduce your topic. Lorem ipsum
ad nauseum. Introduce your topic. Lorem ipsum ad nauseum. Introduce your
topic. Lorem ipsum ad nauseum. Introduce your topic. Lorem ipsum ad
nauseum.

Anything else? Lorem ipsum ad nauseum. Introduce your topic. Lorem ipsum
ad nauseum. Introduce your topic. Lorem ipsum ad nauseum. Introduce your
topic. Lorem ipsum ad nauseum. Introduce your topic. Lorem ipsum ad
nauseum.

Anything else? Lorem ipsum ad nauseum. Introduce your topic. Lorem ipsum
ad nauseum. Introduce your topic. Lorem ipsum ad nauseum. Introduce your
topic. Lorem ipsum ad nauseum. Introduce your topic. Lorem ipsum ad
nauseum.

\section*{Acknowledgements}
\addcontentsline{toc}{section}{Acknowledgements}

These should be included at the end of the text and not in footnotes.
Please ensure you acknowledge all sources of funding, see funding
section below.

Details of all funding sources for the work in question should be given
in a separate section entitled `Funding'. This should appear before the
`Acknowledgements' section.

\section*{Funding}
\addcontentsline{toc}{section}{Funding}

The following rules should be followed:

\begin{itemize}
\tightlist
\item
  The sentence should begin: `This work was supported by \ldots{}' -
\item
  The full official funding agency name should be given, i.e.~`National
  Institutes of Health', not `NIH' (full RIN-approved list of UK funding
  agencies)
\item
  Grant numbers should be given in brackets as follows: `{[}grant number
  xxxx{]}'
\item
  Multiple grant numbers should be separated by a comma as follows:
  `{[}grant numbers xxxx, yyyy{]}'
\item
  Agencies should be separated by a semi-colon (plus `and' before the
  last funding agency)
\item
  Where individuals need to be specified for certain sources of funding
  the following text should be added after the relevant agency or grant
  number `to {[}author initials{]}'.
\end{itemize}

An example is given here: `This work was supported by the National
Institutes of Health {[}AA123456 to C.S., BB765432 to M.H.{]}; and the
Alcohol \& Education Research Council {[}hfygr667789{]}.'

Oxford Journals will deposit all NIH-funded articles in PubMed Central.
See Depositing articles in repositories -- information for authors for
details. Authors must ensure that manuscripts are clearly indicated as
NIH-funded using the guidelines above.


% Bibliography
\bibliographystyle{natbib}
\bibliography{bibliography.bib}

\end{document}
