\documentclass{article}
\usepackage{amsfonts}
\usepackage{graphicx}
\usepackage[margin=1in]{geometry}
\usepackage{bm}
\usepackage{amsmath}
\usepackage{authblk}

\title{Taxonomic Enrichment Analysis with Isometric Log-Ratios}
\author[1,2]{Quang P. Nguyen}
\author[1,2]{Anne G. Hoen}
\author[1]{H. Robert Frost}
\affil[1]{Department of Biomedical Data Science, Geisel School of Medicine at Dartmouth College, Hanover, NH 03755, USA}
\affil[2]{Department of Epidemiology, Geisel School of Medicine at Dartmouth College, Hanover, NH 03755, USA}
\date{}                     %% if you don't need date to appear
\setcounter{Maxaffil}{0}
\renewcommand\Affilfont{\itshape\small}

\begin{document}
\maketitle

\begin{abstract}
    \noindent High-dimensionality is a challenging problem in analyzing microbiome relative abundance data. Studies commonly alleviate this problem by aggregating variables into sets, most commonly higher order taxonomic classifications. However, such approaches are often naive and does not consider the hypothesis aggregation problem when testing for significance at multiple taxonomic levels. Here we introduced a novel competitive taxonomic enrichment method based on the isometric log-ratio transformation (cILR) for single samples. We demonstrated that our method controls type I error and power for hypothesis testing at the single sample level, as well as providing more robust results than other single sample enrichment methods for differential abundance and prediction tasks.   
\end{abstract}

\section*{Background}
\section*{Methods}
\section*{Results}
\subsection*{Simulation Analysis}

\subsubsection*{Type I error control and power}




\section*{Real data analysis} 
\subsection*{Type I error control}
We benchmark type I error rate on 16S and WGS data from the Human Microbiome Project (HMP) obtained from the packages \emph{HMP16SData} (ver. 1.9.3) and \emph{curatedMetagenomicData} packages in R. For each data set, we filtered out samples with library size less than 1000, as well as taxa with a proportion of zeroes of 0.9 or more. We then randomly assigned each sample into one of two arbitrary groups.  \\

\newpage
\bibliography{tax_agg}{}
\bibliographystyle{plain}
\end{document}
\end{document}